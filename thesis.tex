\documentclass[23pt,a4paper]{article}
\setlength{\columnsep}{0.13\columnwidth}
%\setlength{\columnseprule}{0.05\columnwidth}

\usepackage{fontspec}
%\setmainfont{Microsoft YaHei}

\usepackage{bm}
\usepackage{float}
\usepackage{geometry}
\geometry{left=2.5cm,right=2.5cm,top=3.0cm,bottom=2.0cm}
\usepackage{flushend}
\usepackage{fontspec}
\usepackage{xeCJK}
\usepackage{setspace}
\usepackage{subfigure}
\usepackage{listings} %插入代码
\usepackage{float}
\usepackage{amssymb}
\usepackage{amsmath}
\usepackage{xcolor}
\usepackage{amsfonts,color}
\usepackage{graphicx}
\usepackage{booktabs}
\usepackage{longtable}
\usepackage{tabularx}
\usepackage{wrapfig}
\usepackage{indentfirst}
\usepackage{float}								%³¬¼¶ºÃÓ㡸¡¶¯ÅŰ棡
\usepackage{flushend,cuted}
\usepackage{caption}

\captionsetup{font={scriptsize}}						%¸Ä±äͼÃû×ÖÌå´óС

\title{}
\author{\small 刘茁 \footnote{zliupku@pku.edu.cn,北京大学物理学院} }
\begin{document}
%\date{August 5, 2018}
\maketitle
\begin{spacing}{2.0}

\section{绪论}
\subsection{能源与聚变}
地球上的能源分为不可更新能源和可更新能源。前者主要指化石能源,即煤、石油、天然气;后者包括太阳能、水能、风能等。
随着人类社会的发展,人类对能源需求的增长和现有能源日趋减少的矛盾愈发突出。风能、水能、太阳能等能源效率效率有限,
远远不能满足人类的需求,而煤、石油、天然气储量十分有限,据预测,化石能源目前储备量已不足人类使用百年。
另一方面,大量使用化石能源同时也造成了严重的污染,温室效应日渐明显。\par
%table here
在核能源中,核聚变的技术已经相对成熟,然而核聚变原料铀235价格高总量有限,且会产生放射性核废料造成环境污染,更可怕的是时常有
核泄漏的事故发生,要想从根本上解决人类的能源问题还是靠聚变。
核聚变能源是利用热核聚变产生巨大能量的能源。19世纪末,放射性发现之后,太阳能的来源很快地被揭开。英国物理学家阿斯顿利用摄谱仪进行
同位素研究,他发现氦-4质量比组成氦的2个质子、2个中子的质量之和大约小1\%。迄今为止,人类发现最重要的聚变反应是
$$D+D \rightarrow T+ p +4.03MeV$$
$$D+D \rightarrow {}^3He + n +3.27MeV$
$$D+T \rightarrow {}^4He + n + 17.59MeV$$
$$D+{}^3 He \right {}^4He + p + 18.35MeV$$

平均到每核子,聚变能比裂变能大4倍。另一方面,巨变燃料氘资源非常丰富,在海水的氢中,氘占1/6700,可以直接用点解方法从海水中提取氘,其费用与聚变电能的生产相比是微不足道的,
而每升海水中的氘产生的聚变能相当于300升汽油燃烧释放的热量,因此海水中的氘产生的聚变能可供人类使用上亿年\par
实现聚变能的利用,其难度远较裂变能大得多,因为引起聚变反应的两个核都带正电,要使两个核足够接近,必须克服库伦排斥力,
目前的主要实现方法是受控热核反应,如果能将一团氘气体放在容器中加热到足够高的温度并约束足够长的时间,就可以引发聚变反应释放强大的聚变能。目前,人们往往以“点火条件”来估计热核反应堆的能力。直接利用$\alpha$粒子聚变能量加
热等离子体,补充等离子体的辐射和热能损失,使等离子体保持原来的高温状态,继续进行聚变反应,而不需要外界回授
能量或者加热,达到这种条件的聚变堆称为“点火”。想要实现热核反应必须解决一个问题,就是如何约束高温等离子体并达到足够长的时间。\par
目前人类主要想到两个办法,一是惯性约束,即激光核聚变,它利用大功率激光压缩燃料靶丸,依靠粒子自身的惯性在短时间内实现聚变。二是磁约束,即利用磁场约束等离子体。目前来讲,磁约束装置的进展更好,约束时间更长。\par
20 世纪50 年初,算立案物理学家塔姆提出用环形强磁场约束高温等离子体的设想。在环形不锈钢真空室外套有多匝线圈,
利用电容器对多匝线圈放电,使真空室形成环形磁场。与此同时,用变压器放电,使作为次级线圈的环形室中的气体电离,
形成强等离子体电流,此环形电流产生的极向磁场与环形磁场结合,形成约束高温等离子体装置,称为“托卡马克装置”。\par
现在世界上有30多个国家建造了几十个托卡马克装置,包括美国的DIII-D,欧共体的JET和ASDEX,日本的JT-60,中国成都的HL-2A以及合肥的全超导托卡马克EAST等。1991 年11月,JET第一次实现DT聚变反应,
高温3亿度,反应持续2秒。2018年,中国的全超导托卡马克破纪录地实现稳定约束101.7秒,20多年来,等离子体密度,约束时间,燃烧温度三乘积提高了超过10000倍,磁约束聚变实验已经真正进入了氘氚燃烧试验的阶段。\par
2006年11月21日,中国、欧盟、美国、韩国、日本、俄罗斯、印度七方代表在巴黎正式签署了《国际热核聚变实验堆
(International Thermonuclear Experimental Reactor, 简称ITER)联合实施协定》,全面地动了这一人类开发新能源
的宏伟计划。ITER计划的目标是要建造一个可自持燃烧的托卡马克核聚变实验堆,是世界上在建的最大热核聚变堆,
等离子体环大半径为6.2m,小半径为2.0m,氘氚组成的高温等离子体约束的体积达$837m^3$。

\subsection{高能粒子与不稳定性}
在现代托卡马克装置中,等离子的主要成分除了热粒子以外,还包括聚变反应和辅助加热产生的高能量粒子(快粒子)。
一方面,高能量粒子是聚变等离子体的重要能量来源。在DT等离子体中,聚变产生的$\alpha$粒子具有3.5 MeV的能量,将是聚变等离子体自持加热的重要手段。
另一方面,通过辅助加热手段,比如离子回旋共振加热(ICRH)、电子回旋共振加热(ECRH)、中性束注入(NBI)等,也可以产生大量的高能粒子。
高能粒子的能量是热粒子的数十倍甚至跟高,其在等离子体中所占的比例可达到10\%以上,它们可能在速度空间和径向分布
均不均匀。这些高能粒子(EP,energytic particle)深刻影响着装置的运行。高能粒子的损失将会导致大量的能量流失,
影响燃烧效率甚至导致自持燃烧条件无法达到,还可能损坏装置。一方面,背景等离子体的湍流会引起背景等离子体的
径向输运,造成高能粒子的损失。另一方面,高能粒子与等离子体波的相互作用,驱动一系列的不稳定性。
这些不稳定性引起的粒子径向输运同样使装置能量损失。\par
等离子体中高能粒子激发的不稳定性大致可以分为如下几类
\begin{itemize}
\item 高能粒子与扭曲模相互作用而激发的鱼骨模不稳定性
\item 高能粒子与阿尔芬波相互作用而激发的各种阿尔芬本征模(AEs)
\item 高能粒子本身产生的不稳定性,如能量粒子模EPM
\item 高能粒子与其它低频MHD模发生相互作用产生的不稳定性,如动力学气球模(KBM),KBM的激发与压力梯度相关,环向模数n较大。
\end{itemize}
理解这些非线性物理现象的物理机制需要我么对非线性的波-粒相互作用过程进行深入研究。\par

\subsection{BAE & BAAE}
目前,几乎所有托卡马克装置都诊断到了高能粒子激发的不稳定性,AEs一直是研究的重点。首先,AEs由快粒子的径向压强梯度驱动,增强了快粒子的径向输运。其次,AEs可以和快粒子发生强烈的共振,激发不稳定性模式,比如ITER中300-400KeV
的EP将与鱼骨模发生共振。最后,尽管目前检测到的AEs的振幅不是非常大,但是其导致的严重的快粒子径向输运和流失
缺占据了很大一部分。\par
最近在DIII-D 的实验中发现,低频模式造成的高能粒子损失占到了总损失一半,其很有可能是可压缩响应导致的比压型阿尔芬本征模
(Beta-induced Alfven Eigenmode, BAE)以及与声波耦合产生的比压型阿尔芬声波本征模(Beta-induced Alfven Acoustic Eigenmode, BAAE)等低频模式造成的,这些低频模式的激发引起了研究人员的高度重视。事实上,目前有越来越多的装置
发现了BAAE频率的存在(是否确定为BAAE尚存在疑问),包括DIII-D、JET、NSTX、ASDEX。而且BAAE不仅仅存在与低比压
的环境中,在比压大于20\%的环境中也可以被激发。\par
北京大学物理学院刘雅琪博士在前几年利用回旋动理学代码(Gyro-kinetic Toroidal Code, GTC),在圆形截面的解析位型下
对BAE和BAAE进行了模拟,对它们的激发机制进行了深入的研究并提出了创新性的观点。首先,刘雅琪博士利用GTC证实了BAAE
的存在,这与MHD和局域理论的预言是不同的,BAAE在强朗道阻尼下仍然能够存在,这可能意味着BAAE需要一些非局域的理论
来描述。第二,快离子的非微扰效应在BAAE的形成中至关重要。BAE和BAAE在增长阶段都会从快离子吸收能量然后通过朗道阻尼
对热离子做功,快离子的温度、密度梯度等都会对BAAE的特征频率产生影响。第三,当装置尺寸增大时,最不稳定的模式
由BAE向BAAE转化,并预测装置尺寸增大到ITER尺寸时,BAAE的增长率将远大于BAE。

\subsection{BAE}

\section{DIII-D中低频不稳定性的线性模拟}
\subsection{高能离子激发的不稳定性的线性模拟}
DIII-D是1980年代位于美国圣地亚哥的通用原子能公司(General Atomics, GA)在Doublet III装置的基础上建立的一个托卡马克
,其主要目的是为托卡马克的设计优化建立物理基础,其等离子体截面呈"D"型。DIII-D代表了目前世界最先进
磁约束聚变的技术水平,它最早使用中性粒子数注入手段加热等离子体,它可以精确并稳定地在一定范围内控制等离子体的形状和位置,
图中展示了一些MHD可以产生的MHD平衡截面。\par
\begin{figure}[h]%%图
	\centering  
	\includegraphics[width=0.9\linewidth]{./images/3/03.JPG} 
	\caption{DIII-D中可以产生的MHD平衡截面}
	\label{3-01} 
\end{figure}
近几年在DIII-D的实验中,发现了一些低频的不稳定性。前面已经提到了,在低频不稳定性爆发期间,系统损失了约45\%
的加热功率,而低频频率恰处在BAE以及BAAE频率段,这让引起了许多研究人员的密切关注。在第一章中我已经简要介绍了刘雅琪博士在这一方面的工作,
现在我将接着刘雅琪博士的工作,继续使用GTC对快粒子激发的低频不稳定性的线性模拟性模式进行模拟以及研究,并偏重于研究DIII-D中的对应情况。\par
本章中模拟采用的数据来自DIII-D第146094次放电,GTC采用实验平衡数据(EFIT)用ORBIT绘图代码转换成GTC可以使用的输入数据。
这次实验中,最不稳定的环向模数为 n=2,4,6,在 n=2 的高频段,较宽的间隙可能存在离散的TAE 和 RSAE 本征模(约 70~90kHz 等离子体坐标系,或 100~120kHz 实验坐标系),
低频段(10~50kHz)的$\beta$间隙,可能存在离散的 BAE 和 BAAE本征模,恰处于实验中低频不稳定性的频率范围。通过阿尔芬波声波连续谱,可以初步判断本征模所处的径向位置以及频率范围。
\begin{figure}[h]
    \begin{minipage}[t]{0.45\linewidth}
    \centering    
    \includegraphics[width=0.9\textwidth]{./images/3/02.JPG}
    \caption{DIII-D实验EFIT输出的平衡位型和GTC输入的比较}%图片的名称
    \label{3-03-1}%标签,用作
    \end{minipage} 
    \hfill
    \begin{minipage}[t]{0.45\linewidth}
    \centering
    \includegraphics[width=0.9\textwidth]{./images/3/04.JPG}
    \caption{GTC模拟区域示意图,内外环形是场模拟的内外边界,中间环对应$q_min$所在位置}
    \label{3-03-2}
    \end{minipage}
\end{figure}
在这一章所有的模拟中,我都将采用这一次放电的平衡剖面。在这次实验中,有效大半径$R_0 = m$,小半径$a=m$,磁轴处磁场为$B_0 = $,
磁轴处电子温度$ $,电子密度$ $,离子温度,离子密度,快粒子温度,快离子密度,比压值为。安全因子q呈剪切位型,最小值$q_min = 1.44$。
整体模拟所采用背景等离子体的平衡如图所示,快粒子面以及。\par
\begin{figure}[ht]%%图
	\centering  
    \includegraphics[width=0.8\linewidth]{./images/3/07.JPG} 
    \caption{DIII-D \#146094次放电电子、离子、快离子的温度、密度以及安全因子q的剖面图}
	\label{3-04} 
\end{figure}
我们先采用高能粒子(快粒子)在真实实验参数下进行激发,在模拟开始之前,我们先对抗磁漂移频率以及坐标系之间频率关联
作一定的计算。注意GTC模拟是包含抗磁漂移效应($\omega_*)$的,实验室坐标下的频率$\omega_{lab}$与等离子体坐标下的频率$\omega_{plasma}$是有一定差别的,
它们具体的解析关系不能清楚地写出,但是我们可以通过加减$\omega_*$的方式来做简单的估计。\par
考虑热离子和快离子的压强梯度(包括密度梯度和温度梯度效应)
\begin{eqnarray*}
\omega_* = -i\bm{\star} \cdot \nabla  = m(\bm{v_{\star}}\cdot \nabla \theta) \\
\bm{v_{\star}} = \frac{n_{oi}m_i \bf{v_{\star i}} + n_{of}m_f \bf{v_{\star f}}}{n_{oi}m_i + n_{of}m_f}\\
\bm{v_{\star \alpha}} = \frac{c\bm{B_0}\times \nabla P_{0\alpha}}{Z_{\alpha}B_0^2 n_{o\alpha}}, P_{0\alpha} = n_{\alpha}T_{\alpha}
\end{eqnarray*}
\begin{figure}[h]%%图
	\centering  
	\includegraphics[width=0.9\linewidth]{./images/3/08.JPG} 
	\caption{离子、快离子的温度、密度梯度的剖面图}
	\label{3-05} 
\end{figure}
按照上述方法计算,在我们的平衡中,快离子贡献的抗磁漂移占,其中快离子的密度梯度的成分又在快离子中占主要成分,贡献了。
可以看出,在实验位型下,由于快离子面的大梯度,抗磁漂移的频率非常大,在$m=6$的模式中抗磁漂移的频率就与BAAE的频率相当了,
在更高模数的模式中,或者在更大装置尺寸的模拟中,抗磁漂移的频率甚至超过了BAAE的频率,这对我们利用频率的判断造成了非常大
的影响,也对模拟中激发BAAE所在频率的模式造成了一定的困难。\par
在这次模拟中,我对环向模数n进行扫描(n=2,4,6),保留所有纵向模数m(对应GTC中nfilter=1),其最不稳定的纵向模式
分别为$m=3,m=6, m=9$。提取$(2,3),(4,6), (6,9)$这几个模式,得到的频率与增长率结果如图所示。\par

\begin{figure}[h]
    \begin{minipage}[t]{0.5\linewidth}
    \centering    
    \includegraphics[width=1.0\textwidth]{./images/3/05.JPG}
    \caption{(m,n)=(2,3),(4,6),(6,9)三支模线性模拟的频率f,抗磁漂移f*,以及f-f*}%图片的名称
    \label{3-06-1}%标签,用作
    \end{minipage} 
    \hfill
    \begin{minipage}[t]{0.5\linewidth}
    \centering
    \includegraphics[width=1.0\textwidth]{./images/3/06.JPG}
    \caption{(m,n)=(2,3),(4,6),(6,9)三支模线性模拟的圆频率$\omega$,以及增长率$\gamma$}
    \label{3-06-2}
    \end{minipage}
\end{figure}

可以看出来,第一,在线性模拟中我们只得到了单一频率的模式;第二,模拟中最不稳定的模式为n=4,其频率为,增长率为,;
第三,在不同模数的模拟中,得到的频率随纵向模数的增加而增加,我尝试在得到的频率中直接减去每个模对应的抗磁漂移
频率,得到的$f-f*$结果还比较一致。从第三点可以看出,假设不同纵向模式数对应的同一种模式频率差别不太大,
用减法来估计抗磁漂移频率的作用是比较合理的一种方式。\par
我们选取最不稳定的模式,n=4的模式,做出其静电势以及矢势的平行分量的极向以及径向的模结构。
从极向模结构来看,其拥有类似三角形的模结构,但是又与圆位型下的不太一致。这里的模结构三角形没有那么明显,与之前
观测到的RSAE的模结构有一些类似之处。
\begin{figure}[h]
    \begin{minipage}[t]{0.95\linewidth}
    \centering    
    \includegraphics[width=0.9\textwidth]{./images/3/09.JPG}
    \caption{(m,n)=(4,6)在线性模拟中静电势$\phi$模结构}%图片的名称
    \label{3-07-1}%标签,用作
    \end{minipage} 
    \vfill
    \begin{minipage}[t]{0.95\linewidth}
    \centering
    \includegraphics[width=0.9\textwidth]{./images/3/10.JPG}
    \caption{(m,n)=(4,6)在线性模拟中磁矢势平行分量$A_{\parallel}$模结构}
    \label{3-07-2}
    \end{minipage}
\end{figure}
利用MHD代码ALCON绘制阿尔芬-声波连续谱,我们将n=6模模拟得到的频率f和减去抗磁漂移数值的频率同时画在阿尔芬波-声波连续谱中。
\begin{figure}[h]%%图
	\centering  
	\includegraphics[width=0.9\linewidth]{./images/3/11.JPG} 
	\caption{ DIII-D \#146094 环向模数n=4 的阿尔芬波-声波连续谱,其中粗线是阿尔芬波谱,细线是声波谱}
	\label{3-08} 
\end{figure}
我们可以发现,GTC模拟得到的频率落在BAE间隙的上方,靠近RSAE频率,但是减去抗磁漂移的频率后,其落在BAE的区间内。我们初步判定在线性模拟中,
我们得到的是一个单一频率的模,它的频率和特征更靠近于BAE。\par
显然在DIII-D的GTC线性模拟中,我们没有得到BAAE的频率。我尝试了天线扫频。

\end{spacing}
\end{document}

