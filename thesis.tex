\documentclass[23pt,a4paper]{article}
\setlength{\columnsep}{0.13\columnwidth}
%\setlength{\columnseprule}{0.05\columnwidth}

\usepackage{fontspec}
\setmainfont{Microsoft YaHei}

\usepackage{bm}
\usepackage{float}
\usepackage{geometry}
\geometry{left=2.5cm,right=2.5cm,top=3.0cm,bottom=2.0cm}
\usepackage{flushend}
\usepackage{fontspec}
\usepackage{xeCJK}
\usepackage{setspace}
\usepackage{subfigure}
\usepackage{listings} %插入代码  
\usepackage{float}
\usepackage{amssymb}
\usepackage{amsmath}
\usepackage{xcolor}
\usepackage{amsfonts,color}
\usepackage{graphicx}
\usepackage{booktabs}
\usepackage{longtable}
\usepackage{tabularx}
\usepackage{wrapfig}
\usepackage{indentfirst}
\usepackage{float}								%³¬¼¶ºÃÓ㡸¡¶¯ÅŰ棡
\usepackage{flushend,cuted}
\usepackage{caption}

\captionsetup{font={scriptsize}}						%¸Ä±äͼÃû×ÖÌå´óС

\title{}
\author{\small 刘茁 \footnote{zliupku@pku.edu.cn,北京大学物理学院} }
\begin{document}
%\date{August 5, 2018}
\maketitle
\begin{spacing}{2.0}

\section{绪论}
\subsection{能源与聚变}
地球上的能源分为不可更新能源和可更新能源。前者主要指化石能源,即煤、石油、天然气;后者包括太阳能、水能、风能等。
随着人类社会的发展,人类对能源需求的增长和现有能源日趋减少的矛盾愈发突出。风能、水能、太阳能等能源效率效率有限,
远远不能满足人类的需求,煤、石油、天然气储量有限,同时也造成了严重的污染,温室效应日渐明显。\par
\
核聚变能源是利用热核聚变产生巨大能量的能源

\subsection{高能粒子与不稳定性}
在现代托卡马克装置中,等离子的主要成分除了热粒子以外,还包括聚变反应和辅助加热产生的高能量粒子(快粒子)。
一方面,高能量粒子是聚变等离子体的重要能量来源,在DT等离子体中,聚变产生的$\alpha$粒子具有3.5 MeV的能量,将是聚变等离子体自持加热的重要手段。
另一方面,通过辅助加热手段,比如离子回旋共振加热、电子回旋共振加热、中性束注入等,也可以产生大量的高能粒子。
这些高能粒子(EP,energytic particle)影响着装置的运行,一个重要方面就是与等离子体波的相互作用。
在托卡马克中,由于波-粒子共振所激发的阿尔芬本征模(AE,Alfven Eigenmode)、鱼骨模等不稳定性,造成EP的大量损失,影响背景热等离子体的约束。
这些不稳定性引起的粒子径向输运使装置能量损失,还可能导致装置的损毁。\par
目前,几乎所有托卡马克装置都诊断到了高能粒子激发的不稳定性,AEs一直是研究的重点。首先,AEs由快粒子的径向压强梯度驱动,增强了快粒子的径向输运。
其次,AEs可以和快粒子发生强烈的共振,激发不稳定性模式。

\subsection{BAE}

\end{spacing}
\end{document}